\chapter{视觉上下文的感知计算建模}
\label{c2}
视觉上下文工作机制描述着整体感知和局部交互作用的视觉现象,揭示了人类视觉系统表现出自适应性的原因。为模仿视觉上下文的工作机制,就需要全面认识上下文,在把握其内在规律的基础上,建立起生理活动和计算过程之间的联系。受到Marr三层计算视觉框架的设计原则的启发,本章分别将从“人类视觉行为表现”(第\ref{c2:s1}节)、“神经信号处理机制”(第\ref{c2:s2}节)和“计算算法流程”(第\ref{c2:s3}节)三个不同角度{总结和归纳}视觉上下文的工作机制,{提炼}具有普遍意义的规律,{寻求}计算建模的途径。在第\ref{c2:s4}节,{构建了}上下文计算模型,为扩展计算机视觉算法自适应性{提供理论依据}。



\section{视觉上下文的行为表现---格式塔感知机制}
\label{c2:s1}

格式塔(Gestalt)心理学是现代心理学主要的思想流派之一,主要研究的是人类感知行为背后的规律。