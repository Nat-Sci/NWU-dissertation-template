
\begin{abstract}
临床诊断决策依赖于推理等复杂的感知和认知活动:医生首先提出有病假设,然后通过观察影像等信息验证假设;经过多轮推理,在充分消除信息非确定性的基础上,形成最终的诊断结论。相较于目前应用于辅助诊断决策的计算机视觉算法,尽管它们能充分挖掘影像特征和诊断结论之间的映射关系,却还无法像医生一样处理影像中的非确定性信息,进而无法“充分”融入到基于推理的诊断工作流程中。


围绕医生如何消除影像信息非确定性的感知和认知规律这一基本问题,本文开展了系统、深入的研究。考虑到诊断任务的复杂性,本文仅聚焦于“初级视觉感知”规律,具体来说,即视觉系统利用整体信息消除局部信息歧义的上下文工作机制。通过模仿该机制,提升了传统计算机视觉算法自适应性。本文五个研究内容概括如下:



\keywords{上下文计算视觉模型,\quad{} 超像素} \par
\end{abstract}

\begin{englishabstract}



To make a diagnostic decision, doctors rely on a list of perceptual and cognitive behaviors for reasoning. A doctor would first hypothesize a patient with unknown health problems, and then, verify the hypothesis by observing the patient's magnetic resonance (MR) scans. After rounds of inference, the doctors eventually draw the conclusion only if they have eliminated information ambiguity. By contrast, although existing computer-aided diagnosis algorithms can fully mine the correlation between image features and diagnostic conclusions, these algorithms cannot deal with information ambiguity as doctors do. Therefore, they cannot be effectively embedded in the evidence-based diagnosis workflow.


For unveiling how a doctor's visual system eliminates information ambiguity, this dissertation presents a systematical and thorough investigation. Considering the complexity of diagnostic activities, this manuscript only focuses on the primary visual functions of a doctor's individual activities, exploring  how visual context eliminates information ambiguity. What follows,  the author attempts to imitate such mechanisms to build a "computational vision model" in guiding to design "computer vision algorithms" that can adaptively cope with the ambiguity in MR images. The author's five contributions are summarized as follows:




\englishkeywords{Context-based computational vision model, \space{} Superpixel} \par %  Perceptive and cognitive vision model,
\end{englishabstract}

\NWUpremainmatter
