\chapter{总结与展望}
\label{c7}

由于一个大脑皮层区域可能涉及编码多个外部对象,视觉感知系统就不可避免地引入信息歧义或非确定性。基于该理论假设,视觉上下文工作机制阐明了人类视觉系统利用整体信息消除局部信息歧义的感知规律,还揭示了视觉系统产生自适应性能力的原因。本文通过模仿上下文工作机制,提升了三种计算机视觉算法的自适应性。然而,受限于时间和精力,本文的工作只涉及“物理外观”的低层上下文工作机制,在未来的工作中我们再继续讨论对其它三个层面的上下文工作机制建模的思路。



\section{全文总结} 
\label{c7:s1}

本文首次将视觉上下文工作机制引入到计算机视觉领域,构建了上下文计算视觉模型。 这为解决计算机视觉中解不唯一的病态问题提供了一个全新的思路---即从数据中采集上下文信息,以约束局部信息的非确定性。本文主要工作概括如下:



\section{工作展望} 
\label{c7:s2}

受限于时间和精力,本文的局限性也值得注意。它们分别是:



